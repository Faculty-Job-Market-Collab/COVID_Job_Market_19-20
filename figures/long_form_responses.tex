\documentclass[]{article}
\usepackage{lmodern}
\usepackage{amssymb,amsmath}
\usepackage{ifxetex,ifluatex}
\usepackage{fixltx2e} % provides \textsubscript
\ifnum 0\ifxetex 1\fi\ifluatex 1\fi=0 % if pdftex
  \usepackage[T1]{fontenc}
  \usepackage[utf8]{inputenc}
\else % if luatex or xelatex
  \ifxetex
    \usepackage{mathspec}
  \else
    \usepackage{fontspec}
  \fi
  \defaultfontfeatures{Ligatures=TeX,Scale=MatchLowercase}
\fi
% use upquote if available, for straight quotes in verbatim environments
\IfFileExists{upquote.sty}{\usepackage{upquote}}{}
% use microtype if available
\IfFileExists{microtype.sty}{%
\usepackage[]{microtype}
\UseMicrotypeSet[protrusion]{basicmath} % disable protrusion for tt fonts
}{}
\PassOptionsToPackage{hyphens}{url} % url is loaded by hyperref
\usepackage[unicode=true]{hyperref}
\hypersetup{
            pdftitle={Long Form Responses},
            pdfborder={0 0 0},
            breaklinks=true}
\urlstyle{same}  % don't use monospace font for urls
\usepackage[margin=1in]{geometry}
\usepackage{graphicx,grffile}
\makeatletter
\def\maxwidth{\ifdim\Gin@nat@width>\linewidth\linewidth\else\Gin@nat@width\fi}
\def\maxheight{\ifdim\Gin@nat@height>\textheight\textheight\else\Gin@nat@height\fi}
\makeatother
% Scale images if necessary, so that they will not overflow the page
% margins by default, and it is still possible to overwrite the defaults
% using explicit options in \includegraphics[width, height, ...]{}
\setkeys{Gin}{width=\maxwidth,height=\maxheight,keepaspectratio}
\IfFileExists{parskip.sty}{%
\usepackage{parskip}
}{% else
\setlength{\parindent}{0pt}
\setlength{\parskip}{6pt plus 2pt minus 1pt}
}
\setlength{\emergencystretch}{3em}  % prevent overfull lines
\providecommand{\tightlist}{%
  \setlength{\itemsep}{0pt}\setlength{\parskip}{0pt}}
\setcounter{secnumdepth}{0}
% Redefines (sub)paragraphs to behave more like sections
\ifx\paragraph\undefined\else
\let\oldparagraph\paragraph
\renewcommand{\paragraph}[1]{\oldparagraph{#1}\mbox{}}
\fi
\ifx\subparagraph\undefined\else
\let\oldsubparagraph\subparagraph
\renewcommand{\subparagraph}[1]{\oldsubparagraph{#1}\mbox{}}
\fi

% set default figure placement to htbp
\makeatletter
\def\fps@figure{htbp}
\makeatother


\title{Long Form Responses}
\author{}
\date{\vspace{-2.5em}}

\begin{document}
\maketitle

\begin{enumerate}
\def\labelenumi{\arabic{enumi}.}
\item
  Fuck academia - I almost never swear, but seriously, fuck academic.
  Bunch of stuck up pricks!
\item
  Searching for an academic job has been among the most frustrating and
  deeply depressing experiences of my life. I question my self worth
  constantly and never stop worrying that I have wasted my life. I had
  no idea that it would be like this. The actual experience of
  interviewing has been dispiriting. I feel like there is no way to be
  honest and be what search committees want. I find some of them to be
  insulting. One panel, during my 2018 - 2019 search, actually laughed
  at my research. I sometimes wonder why I bother. I loved and still
  love science, but I had no idea what a torment this would be. I feel
  wholly unprepared even now. I was never warned. I have not really felt
  happy once since I started my job search.
\item
  I was pregnant during the entirety of my application and interview
  stages. This definitely had an impact on regional limitations I set
  for applications. I also disclosed my pregnancy so that I could get
  accomodations for an on site interview to be off site during the
  COVID19 shutdown. This job did not out right reject me but said they
  could not extend an offer due to hiring freezes.
\item
  Potential oncampus interviews late in the cycle were indefinitely
  postponed.
\item
  out of my 3 offers, 2 were temporarily frozen/rescinded and then
  reappeared after additional levels of approval. I chose the 1 that was
  never rescinded.
\item
  Na
\item
  Strong publication record and K99 are not ``golden tickets'' to
  academic positions. My lack of success on my first round of academic
  job hunting was humbling. I would have gone for a second round (and
  probably had better success) but covid-19 has interrupted my hope for
  an academic job. I recently accepted a leadership role in an industry
  job to continue my professional development rather than ``waiting this
  out.''
\item
  There are not many resources or hiring initiatives for veterans
  applying for faculty positions
\item
  It has been a very long process which started around Septermber 2019.
  Now it is May 2020, I still have one place that could not make a
  decision due to COVID. I think the brutal job market became even more
  unpredictable and though with COVID.
\item
  I consider the process I went through to have been a valuable training
  exercise. I now have a more firm vision of how I fit into my field and
  what a future faculty position could look like, as well as options
  available to me.
\item
  very stressful, lack of communication
\item
  My experience has driven me to look for jobs outside of the
  tenure-track/traditional academic market
\item
  I really had to be proactive in this process. There wasn't any formal
  support at my institution so I organized my own workshop series on how
  to apply for faculty positions. Without that, I would not have been
  successful.
\item
  I applied with no postdoc papers from my new lab
\item
  I felt that I had a very successful interview season and have ended up
  with no viable job offers. My one open offer is unavailable due to
  COVID19. I am taking a hiatus from applying to jobs because I am
  feeling burnt out from the 2019-2020 job cycle and I am unsure if
  there will be many opportunities for the upcoming cycle.
\item
  It is an extremely grueling process both physically and
  psychologically - by design. I wasn't prepared for how much I would be
  judged as a person as well as a scientist.
\item
  Focusing on applying for academic positions this cycle has likely
  ruined my career in academic science. I have been unable to complete
  new research due to the interview/application process and now
  COVID-19. My supervisor has used this as an excuse not to renew my
  contract in July. Most Universities have issued hiring freezes
  starting now through next year. This leaves me unemployed with no
  academic job opportunities for next cycle.
\item
  I only applied to one position that was close to my husband and my
  current location -- It wasn't a full-blown job search. I'd actually
  prefer to stay as a postdoc for several more years because I am
  switching fields.
\item
  The process of simultaneously building and eroding my confidence
  throughout has been surprising and remarkable. My imposter syndrome
  has never been worse.
\item
  I felt the priorities the job market enforces (number of publications,
  tier of journal) does not reflect the actual tools required for
  successfully running a lab.
\item
  Academia is a literal pyramid scheme
\item
  Visa issue for foreign postdoc -\textgreater{} total dependence to PI
  and institution, easy leading to abusive relationships
\item
  The NIH IRACDA program at my home institution was TREMENDOUSLY helpful
  in preparing me for this process.
\item
  The job search is far from transparent and the lack of communication
  with institutions is a problem. I was asked for an on campus interview
  6 weeks after an initial interview, after accepting another offer.
  Other applications went without any communication at all.
\item
  Variable timing of schools (non-R1s go earlier), difficulty
  understanding non-R1 careers (how much research is actually
  done/valued at R2, R3, SLACs)
\item
  Half of the searches were frozen due to COVID19 before offers were
  made. This impacted the number of offers received and my final choice.
  As a URM, I faced several macro and micro aggressions throughout the
  search in different institutions: from having to show my immigration
  papers to a security guard inside the university to being asked by a
  senior PI the reason why I wanted to stay in the US. As a woman, I
  also received illegal questions about my marital status during the
  search and I was invited for drinks inside the office of a faculty
  member after my interview.
\item
  I was also a `runner-up' at 3 institutions but the covid situation
  created a rush to accept a position due to hiring freezes. I am also
  waiting on the notice of award for a K99 which was scored very highly
  but has been delayed due to covid.
\item
  I, and most of my peers, feel obtaining a Ph.D.~was ultimately a poor
  decision. There are no jobs because baby-boomers are not retiring and
  budgets are shrinking. Nearly all non-academic positions desire MS
  degrees. Institution declarations of supporting diversity are not
  grounded in tangible actions and behaviors. Institutions still seek to
  advance individuals from privileged backgrounds.
\item
  I haven't decided whether or not to accept my offer yet, but feel that
  the anticipated effects of COVID on next year's job market will
  influence my decision
\item
  The pandemic has been a major factor in weakening my committment.
  There are no jobs. The other major factor is racism.
\item
  Covid added so much uncertainty, I received an offer mid-February, and
  it is still being debated now, so I have no idea what is going to
  happen. I am supposed to teach an online class this summer starting
  June 8, and then full-time TT in the fall, but it is all up in the
  air. Which means a move across the country is on hold.
\item
  The tenure track assistant professor position I applied for was no
  longer available once the pandemic hit. I received an email that the
  job search was cancelled indefinitely.
\item
  N/A
\item
  Hiring freezes disrupted the offer process where I had an on campus
  interview
\item
  Yes. Two things: I find the ``good people get jobs'' ideology to be
  totally toxic. So, under question one above, I am ``neutral'' because
  all the adequate preparation I received was poisoned by the discomfort
  (at best) with which my advisors and mentors acknowledged the reality
  of the market. Second, I find the lack of actual feedback from job
  interviews and campus visits to be a serious impediment to
  improvement. How can you work on your job market prep in any rigorous
  way if you have to guess what went wrong?
\item
  I am unemployed (no postdoc, no vap, no adjuncting). I am on medicaid
  and foodstamps. This should be a selection on current employment
  status
\item
  I think each field is different. In my field, educational psychology,
  I have many employment opportunities. As such, it's more of a job
  seeker market in my field.
\item
  My age - completing a PhD in my 50s after a great deal of experience
  in my field (the arts) has been an impediment
\item
  There didn't seem to be a place to record jobs that are still pending
  later due to COVID. I still have two that are still in the process
  with communication of a future on-site visit.
\item
  I'm gonna try again next year but not as hard. It took so much time
  this year. Next year I'll be more selective. I wish I had spent less
  time on it this year and more time on writing (or tbh, fun!). I felt
  well prepared in terms of my written materials but not in terms of
  interview prep and I do not feel I had enough support for that.
\item
  I suffered crushing depression during the entire job market process
  and just as I was emerging (without a job) but beginning to rebuild a
  sense of myself and commitment to try again next year, the COVID-19
  crisis hit and imploded higher education (and the job market with it).
  I feel that any chance of a career in higher education has been lost.
\item
  The most recent position I applied for was frozen due to Covid-19 and
  is not expected to be filled at this point.
\item
  the process is very exclusionary in terms of cost, time commitment,
  and the types of activities you are apparently judged upon
\item
  The PUI job market that I participated in was largely a fall search (I
  submitted apps in Sept/Oct, had remote interviews in Oct/Nov/Dec, on
  campus interviews in Nov/Dec, and job offers + decision made by late
  Dec), so I think somewhat unaffected by COVID. Most PUIs seemed to
  operate their searches on the same fall only timeline, though
  frustratingly there were a few that were delayed by a 1-2 months that
  I had to eventually decline/withdraw my app from because I had to make
  a decision to accept an offer in Dec.
\item
  Due to the response to COVID-19 by universities regarding faculty
  positions and jobs, my ability to remain in my research position and
  publish, I no longer intend to try for an academic position. If I
  stayed in the search and won one, it would not be good enough for me.
\item
  I interviewed with three institutions prior to the COVID-19 outbreak.
  Luckily, nothing was cancelled for me. However, the position that I
  was most interested in was advertised as a tenure track position. I
  phone-interviewed and did well, but just before they began the
  in-person interview, the position was downgraded to a non-tenure track
  position. I regretfully had to reject that job offer. Again, this was
  all prior to the pandemic. So some departments were being tightly
  squeezed even before the fallout from the outbreak.
\item
  Applying for a job in academia is a job unto itself.
  Mentors/supervisors, especially more senior ones, may not see how the
  playing field has changed drastically since they went on the market
  (if they didn't just ascend to the next job without applying), and if
  they do see it, they may not understand how to help the next
  generation deal with the new challenges they face today.
\item
  I have applied for nearly 400 jobs over 4-5 years. In my field, jobs
  go to elite candidates with a few publications over someone who is
  well-published. I regret getting a PhD. I feel I will always be a
  second class citizen because I went to an unranked program. Academia
  is a caste system.
\item
  These questions don't capture the effort that went into applications
  and interviews for searches that were eventually frozen or cancelled
  due to the pandemic.
\item
  I am appalled by the hiring process. I was a finalist for four
  positions over the past two years. One committee filled a position
  with a candidate who has less experience but whose research is only
  vaguely related to the history of the institution's location. In other
  words, the search committee did not select the most qualified
  candidate. Instead, they chose a person whose research is somewhat
  related to the history of the area where the college is located. I do
  not think that candidates should have to submit teaching portfolios,
  diversity statements, or syllabi to secure an interview. The search
  committee should be able to discern if a candidate is a strong fit
  from the letter, CV, and the initial interview. Asking candidates to
  submit additional documents is cruel because of the amount of free
  labor that the practice requires. I am also against asking candidates
  to submit teaching evaluations as proof of teaching effectiveness. It
  indicates that the search committee is completely and deliberately
  ignorant of the studies that prove that students discriminate against
  women, people of color, and non-cis instructors. Finally, campus
  interviews are exhausting, time-consuming, and difficult for people
  with mental illnesses or other health problems. These interviews
  should be shortened to less than a day with no meals. The committee is
  selecting a college to do a job. They are not searching for a soul
  mate or trying to adopt a new family member. Academics need to learn
  how to separate their home life from their business life.
\item
  I graduated in 2012 and still apply when I think I have exact
  qualifications they are looking for, but gave up after yr 3 for active
  search
\item
  Exhaustion
\item
  While I did not receive an official offer, I am in negotiations that
  were delayed due to the current situation, this might be reflected in
  some of my answers
\item
  I still have one ``off-site visit'' scheduled (UMN), so my fingers are
  crossed. I've had more success this year than in any of my previous
  cycles, so I feel very close to the goal. However, the long road to
  this point has been psychologically excruciating and unconscionably
  wasteful of my time on this planet.
\item
  There was little to no feedback from the vast majority of places where
  I submitted applications.
\item
  Still waiting for hiring decision
\item
  I did not realize that universities are run like businesses these days
  and there is such a huge pay gap between different disciplines. Also
  the preferential treatment to top school graduates by top school
  faculty is ridiculous; If one is coming from a medium/low ranked
  school in your discipline, one has almost zero chance of making it to
  a top ranked school even with the required qualifications.
\item
  If the job market has taught me anything it is that hiring choices are
  mercurial.
\item
  As it is currently practiced, the academic job market is not
  sustainable.
\item
  the dual body problem with my partner, who is a postdoc in a related
  field at the same seniority level has been an additional burden
\item
  I was going to say that the h-index of the mentors and institution
  where the applicant is enrolled or got their Ph.D.~from and their
  citizenship status
\item
  The questions are biased towards STEM and missing answer choices. For
  example, I have two solo-authored publications and zero co-authored
  papers; first/last author is meaningless in my field. Also, your
  questions assume the respondent is employed in some capacity; I'm not
  a grad student (I recently defended) but I'm unemployed (so not a
  postdoc, faculty etc).
\item
  the discouraging part of the search was getting verbal offers that
  were rescinded 1 or 2 weeks later. Also, the only remaining soft offer
  has been difficult and slow to negotiate details and get the written
  offer letter.
\item
  I am largely perusing grant funding to support being a research
  associate right now.
\item
  I have decided to no longer pursue jobs in the academic market in the
  near future. May re-evaluate in the next 5-10 years.
\item
  I feel lost and frustrated.
\item
  The difficulty of producing job market materials and customizing them
  to each school makes it difficult to do any other work while applying.
  Which, of course, makes you less likely to get a job the next year.
\item
  Disability status on the job market
\item
  I got lucky. Standard application documents are terrible for assessing
  candidates. Let us know when we're out of the realistic running.
\item
  My case is more complicated. I'm a history PhD who got an MLIS and is
  applying to academic archivist and librarian positions at schools
  where the archivists are part of the faculty association. My PhD
  supervisor is interested and engaged, but of limited help.
\item
  This sounds selfish, but I know that I deserve a tenure track job
  because I am an excellent scholar, an inspiring and organized
  educator, and I stay very active on campus. Despite everything I've
  done, I have failed to secure a tenure-track line. This has been
  heart-wrenching. I have let myself down because being an academic
  means everything to me. This job is my life, and even though I'm
  prolific and throwing myself behind my professional calling, my work
  isn't getting noticed (enough) by hiring committees. While my resolve
  for my research is amaranthine, I've struggled with suicidal thoughts
  this academic year because I applied for several appointments in which
  I knew I would thrive, but the rejections kept coming.
\item
  I am european and did both postdoc and PhD in europe but wanted to
  move to the US, both because I am excited about the research and
  because my partner who is american. Out of the many job applications I
  sent (\textgreater{}50), I only had 1 remote and no onsite, whereas I
  have had 2 remotes on 2 onsite at top european instutions. Looking at
  the people who received interview offers and formal offers, it does
  seem like the US job market is very self centered (Harvard will
  interview people from Rockefeller and UCSF and vice versa) and not at
  all transparent. Also some of these places actually reached out to me
  to ask me to apply (Yale and Harvard) and I did not even get a remote
  interview!
\item
  Despite what was initially a very successful application process (many
  off- and on-site interviews), I only received verbal or written offers
  at places that were not a good fit for me or my family. I had mentally
  prepared for not getting interviews but I had not prepared for the
  possibility that I wouldn't feel comfortable taking a position once it
  was offered and that had the largest affect on my mental health and my
  personal relationships.
\item
  The constant drumbeat of anti-white and anti-male sentiments,
  expressed particularly on social media, are unbelievably hurtful to
  people like me who are just trying to get by find some job stability.
\item
  Why require SO MUCH work like actual letters if by the first paragraph
  of one's cover letter Search Committee knows what they're looking for?
  Ridiculous waste of time for all letter writers.
\item
  I expected my advisor to be the primary source of information/aid,
  that was not the case. My advisor had never heard of ``diversity
  statements'' even though it's required for newly hired faculty in our
  department.
\item
  I adjusted my immigration status from F1 visa to permanent resident
  before entering the job market. The schools where I received offers
  did not ask about my immigration status until AFTER the offer had been
  extended. Even though I didn't need visa sponsorship, both schools
  stated their willingness to sponsor a work visa if I needed one.
\item
  I am in a postdoctoral fellowship program specifically to help
  postdocs get teaching experience and help get a faculty position
\item
  There is a clear lack of transcultural or global recommendations for
  the academic job market. I'm a US citizen living and currently working
  abroad. The process and the required materials for an academic job in
  my current country are similar, but not the same to the process and
  materials found most often in the U.S. This meant that, in some cases,
  I was discouraged from even applying or inquiring about a position
  because I didn't have all of the materials requested by the hiring
  university/college. I ended up with a temporary contract in my current
  country in part because I didn't feel that I could make a strong
  enough case for myself by US standards. While it would require
  substantial ``buy-in'' from a variety of educational systems, I do
  think there is value to at least exploring some kind of basic or
  common job application for folks moving between countries.
\item
  being in a dual career partnership has been very difficult in finding
  suitable opportunities; also, time spent living in the US was very
  difficult for job search
\item
  Losing all sense of hope and will. No job in my area. It was a mistake
  to do my PhD
\item
  My job offer was modified due to hiring constraints caused by COVID19.
  Though I applied for (and would have received) a tenure-track offer,
  it was modified to a VAP until the freeze is lifted at my institution.
\item
  Was selected by a search committee for a TT position, then the search
  was cancelled due to COVID related financial issues
\item
  The process of applying for faculty positions has made me reconsider
  whether I want to get this job which will require me to advise someone
  in a similar position as me (postdoc on the market) in the future.
\item
  The Canadian research job market compared to other nations is weak.
  The poor options from career progress, terrible wage, lack of job
  security, and ability to only hold a fellowship for 5 years all
  contribute to the crumbling of our research field.
\item
  I was the only candidate asked for an on-campus interview for my
  search. My on campus interview, scheduled for the end of March, was
  delayed due to COVID-19 concerns. It was rescheduled as a Zoom-based
  campus interview for a month later only to be postoponed again due a
  university hiring freeze (position not exempted by Provost) but I was
  assured it would go forward by early summer. As of this week, hiring
  for the position has officially been delayed for an unknown anount of
  time, but possibly long term, due to serious budget conerns at the
  instution.
\item
  Indigenous faculty are needed but not hired even though we apply
\item
  I am not applying for jobs outside the academic market and also post
  doctoral positions and fellowships
\item
  Process is stressful and takes a lot of time from research.
  Transparency is low: my competitors had more papers, but they chose
  me. My competitors recently had a kid. Although I was told I set the
  bar very high after my interview. So the concept of fit remains a
  mystery.
\item
  For people from name brand institutions like mine, the job market is
  about luck not merit.
\item
  The prior questions seemed to imply that we have definitely heard back
  (either acceptances or rejections) by now, but myself and many people
  I know have been sent to hiring purgatory where we haven't heard and
  don't know when we will, because the department has a hire they have
  selected and want to extend, but university hiring freezes prevent
  them from doing so.
\item
  I signed a contract for the position I was offered, but the contract
  may not be authorized for up to a year later due to low enrolement
  numbers.
\item
  Things that are appalling about the academic job market: the lack of
  transparency and silence on hiring processes by hiring committees
  during/after virtual or on-campus interviews; hiring committee finding
  ways to ask about the candidates spouse and family situation during
  informal interactions (lunches, dinners) when they know it's illegal.
\item
  I was contacted by a department that was trying to hire in a hurry
  prior to the covid-19-caused hiring freeze announcement, but the
  university pulled the trigger first and my application was ultimately
  accepted for further consideration.
\item
  Still undergoing the interview process for one position.
\item
  University of Nebraska publishes their criteria used for academic
  applications, which was very helpful for this school. Job ads were
  also sometimes misleading - more than once the advert was for a
  generic position but then became clearly specific within the
  application portal
\item
  That you never hear back that committee's have gone another direction
  from some places--even though you might have had a video or in-person
  interview is something that could be easily changed to make this
  process significantly better, but there are so many applications no
  one gives a shit most the time. I've actually gotten automated
  rejection letters from places HR department 2+ years after I was
  interviewed and this is a better job than the majority of places do.
\item
  Fuck academia
\item
  Hand down the worst experience I've had in academia so far. It's
  awful. I always have to remind myself that academia is not a
  meritocracy.
\item
  after 3 years and 75 applications not a single call back so i quit and
  look for admin assistant job
\item
  It seems like a game when there are public posting which are intended
  for internal applicants.
\item
  \textbar{}I was advised that a reference who is high up has bad
  mouthed me to prevent successful job applications - due to the fact
  that I asked someone junior who was insisting on joint first
  authorship to step up to the plate and at a minimum read the
  manuscript they had not contributed to. This `teacher's pet' then told
  the referre untrue statements about me that were believed. The referee
  told me to find 10 other references to refute hers if I was motivated
  to do so, but to otherwise expect that she would claim I was not
  suitable.
\item
  Sharks, sharks, male sharks
\item
  I believe my success was largely due to collaborations, connections,
  and networking. I was in the right place, at the right time, even
  after completing a lot of very intensive and persistent work. The
  academic job market is a lottery, and your success largely depends on
  who else applies!
\item
  interview has not happened yet
\item
  Nowadays I feel only LGBT and women are being considered for faculty
  positions
\item
  Women are seriously disadvantaged in the academic job market, and
  especially women who are not white but also not of a traditionally
  marginalized ethnic group, such as African-Americans. Asian-Americans
  and, for example, Russian or Jewish women and women with accents who
  are non-native speakers of English are not often accounted for in
  affirmative action hiring processes. Merit, quality publications, and
  relevance to interdisciplinary work as well as policy relevance of
  one's research profile are not enough--- today one must be from a
  traditionally marginalized group and/or be from inter generational PhD
  families often with wealth and contacts. I have won many awards for my
  research and have published a monograph on a peer-reviewed North
  American press in the top 10\% of university presses, but I nearly
  always lose out to the children of professors or to men or to disabled
  or African-American or other traditionally disadvantaged groups. As a
  mother from a working class background who is also the breadwinner for
  my immigrant husband who is from a formerly Soviet country it often
  feels the cards are stacked against us and we should give up instead
  of accepting one more postdoc where I have published twice as much as
  my ``supervisor'' who does not even answer email more than once per
  month. The system is in serious need of reform because women and
  others like me who have only talent and merit to compete with are very
  close to contributing to the brain drain by giving up.
\item
  The questions did not reflect asking about the number of cancelled
  searches since COVID-19. Several of the searches still in process in
  the past month or so have been cancelled. This directly affects my
  ability to get a job and stay in academia.
\item
  I think that you should also ask how many jobs have you come across
  that you would be eligible for or do you feel that there is a call for
  your research area - I would apply for a lot more but that area of
  health sciences hasn't had many posts!
\item
  As a humanities student, the Google Scholar questions (especially
  relating to my supervisors) aren't really relevant or helpful. For
  example, my PhD supervisor hasn't set up a Google Scholar, but she is
  a major contributor to our field.
\item
  Many of my offers are still up in the air because of COVID-19, so
  these may not be final/representative statistics.
\item
  There is so much chance involved that it sometimes seems arbitrary and
  that the position I got was due to some extent to luck!
\item
  the 1 ``on campus'' interview i had lined up was cancelled as the
  department decided to hire no one due to expected cuts after covid-19
\item
  I intentionally did not try very hard during my first year of
  applications; I was ``testing the waters''
\item
  My lack of a meaningful relationships while I'm still single is
  somewhat starting to impede my professional development.
\item
  I am very concerned about the impact of Covid19 and now strategizing
  for all types of jobs given that academic jobs will be extremely
  limited.
\item
  Need for current and previous supervisors to provide more guidance;
  Recognize and make students/postdoc on the job market realize its not
  a meritocracy and that there is more to job search than just strong
  statements and resume
\item
  I did not realize how savage the world of Academia was. After being
  encouraged to apply to 2 full time positions within the department
  that I serve as an Adjunct Faculty, I was told that I would not be
  invited to interview for either full time teaching position because my
  PhD was not in the same discipline as my department. I was told that
  the majority of the search committee was comprised of traditional
  professors and they were not interested in having interdisciplinary
  faculty. They did say that I could remain an adjunct faculty member.
  So in other words, I was good enough to teach 70 students each
  semester making a non-living wage, but not good enough to teach the
  same students as a full time faculty member. It was then that I
  decided to not pursue any other academic positions--especially since I
  am land-locked and the competition is very steep. I am currently
  searching for research industry jobs.
\item
  The career development director for postdocs at Emory University was
  incredibly helpful during this time. My primary postdoc mentor was
  completely MIA and unhelpful during this process, so I relied on my
  PhD advisor. Because of COVID, I don't feel comfortable moving away
  from my family and partner to an unfamiliar part of the country so I
  turned down a tenure track offer. That was also because I am queer and
  was being asked to move to a rural Southern area where I would have
  been isolated and less supported (I noticed this survey did not ask
  for sexuality identity which will be a gap in your analyses).
\item
  Positions I applied for were not filled due to hiring freezes and
  pauses associated with the COVID-19 crisis
\item
  It's a rough market out there, and it doesn't seem like it's going to
  get better in time for me to avoid being `spoiled fruit'; Honestly, if
  the regular job market wasn't also terrible at the moment, I'd
  probably look at making the switch now.
\item
  Just a growing concern over how Covid-19 will impact it further. Also,
  an unfortunate consequence of being a postdoc at a leading institution
  is the policy that prevents postdocs from serving as PIs on grants, so
  grants I have helped write or written exclusively that have been
  funded give no credit to me directly.
\item
  14 positions I applied to were closed due to Covid-19 before offers
  were made.
\item
  I was informed that I would have received an offer but due to the
  pandemic they were not able to move forward
\item
  All of my mentors have been so OLD (got their faculty positions in the
  60's/70's) and totally unable to advise how to navigate modern
  academia. None of them has even bothered to get a Google Scholar
  Profile, and they stopped applying for grants decades ago.
\item
  During first job cycle and second job cycle, I sought therapy for the
  first time in my 37yrs of life. I also got extra babysitting help
  (4hrs/week) for a year to give me more time to work on job application
  and project, etc.
\item
  Don't understand why a big public R1 university refuses to a real
  reason or to communicate why my draft offer cannot be further
  proceeded. In my opinion a deny from Provost doesn't explain anything,
  because the budget is allocated and the overall endowment looks pretty
  good. I offered to decrease my startup package and defer my offer, but
  it was an immediate no. From how quick the response I got, there was
  obviously no faculty meeting or anything. The chair was the only
  person to communicate with the university, but he doesn't want to talk
  and was rude when I worked hard for making the offer real. There is a
  huge lack of ways to get information.
\item
  Pretty sure I'm quitting the academic job market
\item
  I also applied to non-academic positions, which I found much
  easier/transparent. I ultimately accepted a position as a research
  scientist at a non-academic institution to do related research.
\item
  I was a breastfeeding mom when I conducted all my interviews, which
  brought me additional challenges. The level of support varies from
  institution to institution.
\item
  I received a job offer in the 2018-2019 cycle at a PUI but did not
  accept it. I had three interviews in that cycle, but none in the
  current cycle. The difference was I previously applied to PUIs,
  whereas this cycle I applied to mostly R1s.
\item
  With my interview cancelled due to COVID and the clear downward
  trajectory of finances of academic institutions, I've decided to leave
  academia
\item
  The position I interviewed for on-site at Harvard was cancelled due to
  Covid-19 and so I felt very disappointed to have wasted so much energy
  on that application and interview process. In general, it's extremely
  frustrating not to be able to get feedback from anyone other than very
  vague comments. University of Norway (I applied last year and my
  friend was in the final 3 applicants) has an excruciantingly
  transparent process which takes forever but takes some of the sting
  out of rejection by at least giving very details justifications.
\item
  My job search was pre-COVID, mostly fall 2019
\item
  My offers are contingent on my receiving a K which is very stressful.
  Of note, I'm still in the process of deciding on which offer to take
  but that wasn't an option.
\item
  I had one offer that was rescinded early March (right before Covid)
  during the negotiation process. I have counted that as a rejection.
\item
  I feel like I have not been able to enjoy my postdoc, even though it
  is to work on a project I am excited about, due to the stress and time
  commitment of going on the academic job market.
\item
  I was surprised how often I was asked illegal questions about marital
  status and children, especially by senior faculty (on the search
  committee!) who should know better
\item
  State budgetary issues and university hiring freezes have me very
  pessimistic for my academic career in coming job searches
\item
  May universities require/expect personalized refenece letters and it
  is very stressful for me to ask each referees to do so much work -
  write and send so many different letters. Pushing for these letters
  was the worst part of the job search.
\item
  Rough experience, especially mentally. Not clear why the organizations
  are not even able to communicate with all candidates, we aren't
  applying to McDonald\ldots{}.
\item
  At one institution I was offered an on-site interview but declined it
  because I learned from the off-site interview and speaking to a former
  faculty member in the department that it would not be a good fit for
  me.
\item
  It's a slog. It's a given that you'll be qualified and prepared -- but
  then you have to be lucky, too, to get an offer.
\item
  As an URM applicant the application process felt particularly
  disheartening. Hard to read the departamental climate from websites,
  coupled with imposter syndrome, meant I may have not applied to places
  that ultimately could have been a good fit.
\item
  I accepted a position in Dec 2019 so the COVID-19 pandemic did not
  have an effect on my job search.
\item
  In my field (molecular biology \& genetics) it is common now to see
  new hires at lower-middle tier schools have first author CNS papers
  along with other supporting high impact papers. Throughout my PhD and
  postdoc, everyone stressed the importance of getting a K99, which I
  did, but in retrospect I wish I had focused more on high impact
  papers.
\item
  Sexual orientation. Also some questions assume I am a postdoc even
  though I indicated being a research associate
\item
  Current position does not have an advisory. I am an independent
  non-tenure track faculty in a host lab. For advisor specific questions
  I answered using my post-doc advisor.
\item
  Faculty career development programs at a large R1 did not prepare me
  adequately for a teaching position search, contact with faculty at
  PUI/SLAC institutions was much more helpful.
\item
  I significantly underestimated how much of My time would be consumed
  by application and interview prep!
\item
  I think if, due to future cuts, I were to have to find myself
  unemployed, I'm not sure I'd return to the academic job market.
\item
  my interview that resulted in an offer was may 29, and offer 2 weeks
  later. Their search was delayed due to covid.
\item
  Emphasis is on quantity over quality. Many people in lab that put
  everyone on every paper are being hired despite being weaker
  candidates in all other areas
\item
  I was successful, but the process was opaque and confusing and I felt
  that everything was a secret I needed to figure out.
\item
  Visiting assistant professors have limited advising/supervision when
  on the academic job market, and do not really get support from their
  institution while they are applying.
\item
  The effect of Covid has been really extreme, with holds and freezes
  impacting my offers.
\item
  Left academia after getting a permanent scientific job for the federal
  governent
\item
  Due to COVID, my responses to the job search etc. for the 2019-2020
  year are irrelevant now. None of this applies when there are few to no
  jobs available, as universities are wisely going mostly or all online
  next year and dealing with major budget (read: hiring) issues.
\item
  I have applied to 400+ jobs over 5 years. In my field, almost all of
  the jobs go to candidates from elite pedigrees. Departments would
  rather hire an unproductive candidate with pedigree than someone like
  me. My PhD and dozens of publications are borderline worthless.
\item
  Stop asking for tailored materials from applicants. Everyone knows
  they're just tweaking what they've got. Standardized documents now!
\item
  It feels like there could be a more centralized application process
\item
  no
\item
  The offer I received and accepted is for a visiting position, not
  tenure track
\item
  After a certain point I began to apply for non-academic jobs alongside
  academic jobs, in part just so I could claim income and in part
  because I was sick of the rat race. It was such a discouraging
  experience that I see other postdocs - peers in the same position as
  me - and want so badly to be able to help. The feeling of getting a
  job when others are striving for the same hurts and makes me feel a
  sort of imposter syndrome where I don't deserve the job I am
  transitioning to.
\item
  I was interested in positions without research responsibilities and
  only teaching/service positions, and while these are difficult to find
  and obtain under normal circumstances, I believe it was more difficult
  to find and obtain such positions given the pandemic.
\item
  I wish there was a set timeline. I felt so anxious just waiting for
  any sort of response.
\item
  I am geographically limited to follow my husband, who is searching for
  a tenure-track R01 position (whereas I am searching for a primarily
  teaching position). We are also looking internationally, which
  drastically reduces the available options.
\item
  There is a lot of hidden knowledge that is shared within networks, and
  it really seems to matter for who gets interviews. Much of the advise
  I received from URM mentors and less from my advisors
\item
  Recently obtained a position outside of academia (bio pharmaceutical
  industry)
\item
  I felt that there were parts of the search that I was not able to
  anticipate due in part to lack of transparency. However, I felt that
  my interactions were fruitful at all the places I interviewed, and
  although I did not receive offers from varies institutions I have
  broadened my network of colleagues. The places I didn't get offers
  from I also felt that the fit was not right so I was not entirely
  disappointed.
\item
  I felt desperately under-prepared for negotiations. I feel like I lost
  leverage due to COVID19.
\item
  I was pregnant for all of my interviews, although not noticeably so
  and I did not disclose (it was either early or hidden by Zoom). 3 of
  my interviews were a result of my husband's offers, although they did
  not result in anything due to bad fit or Covid freezes. Based on our
  experience even schools that claim to have good spousal hiring
  policies do not have an ability to create tenure track lines and will
  not be as accommodating as advertised.
\item
  I was more casually applying for academic positions in the past cycle
  because I was in the first year of a T32 fellowship at the time and
  leaving wasn't really my priority with the payback agreement.
\item
  I was at an institution with few postdocs and no postdoc support. I
  really had to take matters into my own hands to create a job app
  workshop and find community on FuturePIslack. Without those things, I
  would not have been successful.
\item
  I found the peer support groups (especially anonymous ones) incredibly
  helpful.
\item
  I am moving my goals from academic to industry career.
\item
  Couple of pages back on offers -- all choices were accepted or
  rejected. No choice for my current, still in negotiations. Covid
  delayed the cycle.
\item
  it's tough out there
\item
  Lack of transparency is difficult here because of how we all commit
  our lives to this. Some feedback would be extremely helpful.
\item
  The over abundance of part-time or adjunct positions compared to
  full-time or tenure-track positions is shocking.
\item
  Recommendation letters should never be requested upon application
  submission. They should only be requested if you are in fact a top
  candidate. Let's not tax the entirety of academia while us postdocs
  apply for 10+ jobs per year.
\item
  I feel it is unequal. Personal connections have the primary importance
  and competency as a researcher ranks second or lower\ldots{}
\end{enumerate}

\end{document}
